\documentclass[11pt]{article}
\usepackage{newcent}
\usepackage[english]{babel}
\usepackage[letterpaper]{geometry}

\begin{document}

\def\mystrut{\raise-5pt\hbox{\rule{0pt}{20pt}}}
\begin{centering}
\noindent{\Large\bf QPlot: Publication Quality 2D Graph\mystrut\\
  With Less Manual Effort \mystrut\\
  Due to Explicit Use of Dual Coordinate Systems\mystrut}\bigskip

\noindent{\large\bf Tutorial Introduction}\bigskip

\noindent{\Large\bf Daniel Wagenaar, 2013}\bigskip

\end{centering}

\noindent This document serves as a brief introduction to QPlot, the
software described in (Wagenaar, 2013). It is assumed that the reader
has basic familiarity with the Matlab language: While QPlot can be
used as a stand-alone program, it is generally much more convenient to
use it from within a general computation environment such as Matlab or Octave.

\end{document}
