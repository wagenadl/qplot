\documentclass[11pt]{article}
\usepackage{newcent}
\usepackage[english]{babel}
\usepackage[letterpaper]{geometry}

\def\cmd#1#2{\noindent {\bf #1} #2\par}
\def\expl#1{\kern-8pt\begin{itemize}\item[]#1\end{itemize}}
\def\cref#1{{\bf #1}}
\def\bar{{$|$}}
\def\nyi{{\par\bf\itshape Not yet implemented.}}
\def\matlab#1#2{{\bf #1}: #2\par}

\begin{document}
\section{Commands}
Optional arguments are given in parentheses. Vertical bars indicate
alternatives.\bigskip

\cmd{align}{left\bar{}right\bar{}center\bar{}top\bar{}bottom\bar{}middle\bar{}base}
\expl{Sets horizontal and/or vertical alignment for subsequent text.}

\cmd{area}{[ \emph{dx$_1$ dx$_2$ \ldots} ] [ \emph{dy$_1$ dy$_2$ \ldots}
]}
\expl{Draws a polygon, filled using the current brush. 
  \emph{dx$_i$} and \emph{dy$_i$} are specified in
  points and are relative to the position set by \cref{at}. Note that
  the ``['' and ``]'' are literal brackets that separate the x- and 
  y-coordinates. See also \cref{patch}.}

\cmd{at}{\emph{x} \emph{y}}
\cmd{at}{\emph{x} \emph{y} \emph{$\xi$} \emph{$\eta$}}
\cmd{at}{\emph{x} \emph{y} \emph{$\phi$}}
\cmd{at}{--}
\expl{Places subsequent text and lines at graph
    position (\emph{x}, \emph{y}). If (\emph{$\xi$}, \emph{$\eta$})
    are given, this specifies a rotation such that the baseline of the
    text is in the direction of the vector (\emph{$\xi$},
    \emph{$\eta$}). This vector is specified in data
    coordinates. Alternatively, a rotation may be specified as a
    direct (clockwise) angle $\phi$. Besides a numeric value, \emph{x}
    may be one of {``left''}, {``right''}, or {``center''} to place 
    relative to the
    bounding box of the last drawn object (or group, see
    \cref{group}), or a dash (`--') to revert to absolute placement in
    the horizontal direction.  
    Likewise, \emph{y} may be one of {``top''}, {``bottom''}, {``middle''}, or a dash. ``at --''
    reverts to absolute placement (relative to top left of the
    figure (not the current panel!))
    in both horizontal and vertical directions.}

\cmd{brush}{(\emph{ID}) \emph{color}\bar{}none\bar{}\emph{opacity} \ldots}
\expl{Selects a brush by \emph{ID}, defines its color (or sets it to
  ``none''), and/or its \emph{opacity} (as a number between 0 and 1).}

\cmd{figsize}{\emph{w} \emph{h}}
\expl{Sets the size of the figure to (\emph{w} x \emph{h})
  points. This should appear before any other commands.}

\cmd{font}{\emph{family} (bold) (italic) \emph{size}}
\expl{Selects a new font with a given family, point size, weight
  and/or slant.}

\cmd{fudge}{(\emph{margin})}
\cmd{fudge}{\emph{margin} \emph{ratio}}
\cmd{fudge}{-- \emph{ratio}}
\expl{Shrinks the axes as necessary so that all graphics and text fits
  within the bounding box of the figure as defined by
  \cref{figsize}. Margin is specified in points. Optional \emph{ratio}
specifies the desired aspect ratio of y:x data units.}

\cmd{garea}{( \emph{ptspec} ) \ldots}
\cmd{gline}{( \emph{ptspec} ) \ldots}
\expl{Ultraflexible polygon and line series drawing. 
  Each vertex is specified by a \emph{ptspec}, i.e., a sequence of one or
  more subcommands:\medskip\\
\mbox{}\kern10pt\begin{tabular}{lp{3.8in}}
{\bf absdata} \emph{x} \emph{y} & Absolute data coordinates \\
{\bf reldata} \emph{dx} \emph{dy} & Relative data coordinates \\
{\bf abspaper} \emph{x} \emph{y} & Absolute paper coordinates (in pt)\\
{\bf relpaper} \emph{dx} \emph{dy} & Relative data coordinates (in
               pt)\\
{\bf rotdata} $\xi$ $\eta$ & Rotate by atan2($\eta$, $\xi$) 
              in data space (this affects subsequent relative
              positioning) \\
{\bf rotpaper} $\phi$ & Rotate by $\phi$ radians (this affects
subsequent relative positioning) \\
{\bf retract} \emph{L} & Retract preceding and following segments by
              \emph{L} pt \\
{\bf retract} \emph{L$_1$} \emph{L$_2$} & Retract preceding and following
              segments by \emph{L$_1$} and \emph{L$_2$} pt respectively \\
\end{tabular}\medskip\\
Note that the parentheses are literal, unlike in the rest of this manual, where
they designate optional parameters. For instance:\medskip\\
\mbox{}\kern15pt
       gline ( absdata 0 1 relpaper 5 0 ) ~ ( absdata 0 1 relpaper 0 5 )
\medskip\\
     draws a line from 5 pt to the right of the point (0,1) in the graph to
     5 pt above the point (1,0) on the graph.\\
(Note: The rather cumbersome syntax of \cref{gline} makes \cref{line}
     and \cref{plot} more attractive for general usage. The same
     applies to \cref{garea} versus \cref{area} and \cref{patch}.)
}


\cmd{group}{}
\cmd{endgroup}{}
\expl{Groups statements to accumulate bounding boxes for
  \cref{at}. \cref{endgroup} also restores pen, brush, alignment,
  font, and
  reference text to their states before the corresponding
  \cref{group}. Note that named pens and brushes changed inside a
  group are not restored. All groups must be closed before changing
  panels, else the group stack is cleared automatically. }

\cmd{hairline}{1\bar0}
\expl{Enables or disables automatic drawing of zero-width hairlines
 over other lines. Default: disabled.}

\cmd{image}{\emph{x y w h K} [ \emph{cdata} ]}
\expl{Renders an RGB image at given data location. \emph{cdata} is
  stored as (R,G,B) pixels in row order (unlike matlab's convention); 
   \emph{K} specifies
  the number of pixels per row. The length of \emph{cdata} must be an
  even multiple of 3\emph{K}. Values must be between 0 and 1.}

\cmd{line}{[ \emph{dx$_1$ dx$_2$ \ldots} ] [ \emph{dy$_1$ dy$_2$ \ldots} ]}
\expl{Draws a polyline. \emph{dx$_i$} and \emph{dy$_i$} are specified in
  points and are relative to the position set by \cref{at}. See also
  \cref{plot}.}

\cmd{panel}{\emph{ID}\bar--}
\cmd{panel}{\emph{ID x$_0$ y$_0$ w h}}
\expl{Defines a new panel with given ID to have its top left corner on
  paper position (\emph{x$_0$}, \emph{y$_0$}), in points, and size
  (\emph{w} x \emph{h}), in points. Or, reenters a previously defined
  panel. Or drops out to the top level. While drawing inside a panel,
  \cref{figsize} changes the size of the panel, and \cref{fudge}
  affects the panel rather than the figure as a whole. Also
  \cref{xaxis} and \cref{yaxis} affector the axes in the
  panel. Choices of pen, brush, font, etc., are not local to panels.}

\cmd{patch}{[ \emph{x$_1$ x$_2$ \ldots} ] [ \emph{y$_1$ y$_2$ \ldots} ]}
\expl{Draws a polygon, filled using the current brush. \emph{x$_i$}
  and \emph{y$_i$} are specified in data coordinates. See also \cref{area}.}

\cmd{pen}{(\emph{ID})
  \emph{color}\bar{}\emph{width}\bar{}miterjoin\bar{}beveljoin\bar{}roundjoin\bar{}flatcap\bar{}squarecap\bar{}roundcap\bar{}solid\bar\linebreak\mbox{}\kern25pt{}dash\bar{}dot\bar{}dashdot\bar{}dashdotdot\bar{}none
\ldots}
\expl{Selects a pen by \emph{ID}, defines its color, its width (in
  points), its join style, its cap style, and/or its dash
  pattern. Setting the color or width while the dash pattern is
  ``none'' automatically switches to ``solid.''}

\cmd{plot}{[ \emph{x$_1$ x$_2$ \ldots} ] [ \emph{y$_1$ y$_2$ \ldots}
]}
\expl{Draws a polyline. \emph{x$_i$}
  and \emph{y$_i$} are specified in data coordinates. See also
  \cref{line}.}

\cmd{reftext}{\emph{string}\bar{}--}
\expl{Sets or unsets a fixed text that will be used to calculate the
  ascent and descent of text for the ``bottom'' and ``top'' alignment
  modes.}

\cmd{text}{\emph{dx} \emph{dy} \emph{string}}
\expl{Places the given text \emph{string} at the position (\emph{dx},
  \emph{dy}), specified in points relative to the anchor set by
  \cref{at}.\\
  Underscores and hats  make
  subscripts and superscripts (up to the next space). Slashes and
  asterisks make enclosed words appear in /italics/
  and *bold*. Unicode is supported. The string must be enclosed in single or double quotes.}

\cmd{xlim}{\emph{x$_0$} \emph{x$_1$}}
\expl{Fixes the limits of the x-axis. If no xlim is given, tight
  automatic axis limits apply.}

\cmd{ylim}{\emph{y$_0$} \emph{y$_1$}}
\expl{Fixes the limits of the y-axis.}

\section{Specifying data}

For specifying very long vectors or image data, the text-based ``[ a b c ... ]''
syntax may be cumbersome. Instead, you can write ``*\emph{n}'' and
place direct binary doubles after the command. Or you can write
``*uc\emph{n}'' and place direct binary uint8s after the command. The
matlab functions \cref{qplot} and \cref{qimage} give examples.

\section{User interface}

QPlot can be run on the command line, like this:
\begin{quotation}
qplot \emph{source} \emph{output}
\end{quotation}
 where \emph{source} is a file with commands and
\emph{output} can specify either pdf, svg, png, or tiff output. Output
can also be a postscript (.ps) file, in which case a single full page
is produced with the graph in the middle and crop marks around
it. Output to eps is not directly supported, but ``pdftoeps -eps -level3'' can
be used as a postprocessor.

QPlot can also be run interactively, like this:
\begin{quotation}
qplot \emph{source}
\end{quotation}
 In that case, graphics are rendered in a window. Keys ``+''
and ``-'' zoom in and out, ``0'' scales to fit, ``1'' scales to
100\%. ``E'' resizes the window to fit the whole scene. ``G'' toggles
between white and gray borders. ``C'' enables or disables reporting
coordinates below the mouse pointer. ``Ctrl-Q'' quits. The graphics are
automatically rerendered if the \emph{source}
file changes on disk.

A convenient set of Matlab/Octave functions are provided to interact
with QPlot as well:\medskip

\matlab{clq}{Clear current QPLOT figure}
\matlab{qalign}{Set alignment for following text}
\matlab{qarea}{Draw a polygon in paper space}
\matlab{qarrow}{Draw an arrowhead}
\matlab{qat}{Specify location for future text}
\matlab{qbrush}{Set brush for QPLOT}
\matlab{qclose}{Close a QPLOT window}
\matlab{qcolorbar}{Adds a colorbar to the figure}
\matlab{qendgroup}{Ends a group for bbox collection}
\matlab{qerrorbar}{Draw error bars}
\matlab{qerrorpatch}{Draw error patch}
\matlab{qfigure}{Open a QPLOT figure}
\matlab{qfont}{Select font }
\matlab{qfudge}{Add margin to QPLOT panel}
\matlab{qgroup}{Starts a group for bbox collection}
\matlab{qimage}{Plot an image}
\matlab{qimsc}{Plot 2D data as an image using lookup table}
\matlab{qline}{Draw a line series in paper space}
\matlab{qlut}{Set lookup table for future QIMSC.}
\matlab{qmarker}{Select a new marker for QPLOT}
\matlab{qmark}{Draw on the current graph with the current marker}
\matlab{qmticks}{Add more ticks to an existing axis}
\matlab{qpanel}{Define a new subpanel or reenter a previous one}
\matlab{qpatch}{Draw a polygonal patch in data space}
\matlab{qpen}{Selects a new pen for QPLOT}
\matlab{qplot}{Draw a line series in data space}
\matlab{qprint}{Print current QPLOT figure to the default printer}
\matlab{qreftext}{Set reference text}
\matlab{qsave}{Saves a qplot figure}
\matlab{qselect}{Select a QPLOT figure by name}
\matlab{qskyline}{Skyline plot (bar plot)}
\matlab{qsubplot}{Define a new subpanel in relative units}
\matlab{qtextdist}{Specifies distance to text labels for QXAXIS and QYAXIS}
\matlab{qtext}{Render text }
\matlab{qticklen}{Specifies length of ticks for QXAXIS and QYAXIS}
\matlab{qtitle}{Render a title on the current QPLOT}
\matlab{qxaxis}{Plot x-axis}
\matlab{qxlim}{Set x-axis limits}
\matlab{qyaxis}{Plot y-axis}
\matlab{qylim}{Set y-axis limits}\medskip

 Documentation on any of these can be obtained using ``help''
within Matlab or Octave.
\end{document}
